%% LyX 2.3.6.1 created this file.  For more info, see http://www.lyx.org/.
%% Do not edit unless you really know what you are doing.
\documentclass[oneside,english]{book}
\usepackage[T1]{fontenc}
\usepackage[latin9]{inputenc}
\setcounter{secnumdepth}{3}
\setcounter{tocdepth}{3}
\usepackage{amsmath}
\usepackage{amsthm}

\makeatletter
%%%%%%%%%%%%%%%%%%%%%%%%%%%%%% Textclass specific LaTeX commands.
\usepackage[margin=1in]{geometry}
\usepackage[lined,boxed,ruled,norelsize,algo2e]{algorithm2e}

\theoremstyle{plain}
\newtheorem{thm}{Theorem}[chapter]
\newtheorem*{thm*}{{Theorem}}
\newtheorem{cor}[thm]{Corollary}
\newtheorem*{cor*}{Corollary}
\newtheorem{lem}[thm]{Lemma}
\newtheorem*{lem*}{Lemma}
\newtheorem{prop}[thm]{Proposition}
\newtheorem*{prop*}{Proposition}
\newtheorem{conjecture}[thm]{Conjecture}
\newtheorem*{conjecture*}{Conjecture}
\newtheorem{fact}[thm]{Fact}
\newtheorem*{fact*}{Fact}

\theoremstyle{definition}
\newtheorem{defn}[thm]{Definition}
\newtheorem*{defn*}{Definition}
\newtheorem{example}[thm]{Example}
\newtheorem*{example*}{Example}
\newtheorem{problem}[thm]{Problem}
\newtheorem*{problem*}{Open Problem}
\newtheorem{xca}[thm]{Exercise}
\newtheorem*{xca*}{Exercise}
\newtheorem{sol}[thm]{Solution}
\newtheorem*{sol*}{Solution}

\theoremstyle{remark}
\newtheorem{rem}[thm]{Remark}
\newtheorem*{rem*}{Remark}
\newtheorem{idea}[thm]{Idea}
\newtheorem*{idea*}{Idea}
\newtheorem{claim}[thm]{Claim}
\newtheorem*{claim*}{Claim}


\usepackage{titlesec}
\usepackage{fancyhdr}
\pagestyle{fancy}
\usepackage{fix-cm}
\makeatletter
\newcommand\HUGE{\@setfontsize\Huge{38}{47}}
\makeatother

\titleclass{\part}{top}

\titleformat{\part}[display]
  {\normalfont\HUGE\sffamily}{\partname\ \thepart}{0pt}  
  {\titlerule\vskip2pt\titlerule\vskip20pt\HUGE\bfseries\filleft}
  
\titlespacing*{\part} {0pt}{30pt}{50pt}

\titleformat{\chapter}[display]
  {\normalfont\LARGE\sffamily}{\chaptertitlename\ \thechapter}{0pt}  
  {\titlerule\vskip2pt\titlerule\vskip20pt\LARGE\bfseries\filleft}

\titlespacing*{\chapter} {0pt}{20pt}{30pt}

\titleformat{\section}
  {\normalfont\Large\bfseries\sffamily}{\rule[.12ex]{8pt}{8pt}~\thesection}{0.5em}{}
  
\titleformat{\subsection}
  {\normalfont\large\bfseries\sffamily}{\rule[.12ex]{8pt}{8pt}~\thesubsection}{0.5em}{}
  
\fancyhf{}
\fancyhead[ER]{\footnotesize\sffamily\leftmark}
\fancyhead[OL]{\footnotesize\sffamily\nouppercase\rightmark}
\fancyhead[EL,OR]{\bfseries\thepage}
\usepackage{tikz}

% Define scaletikzpicturetowidth
% See https://tex.stackexchange.com/questions/6388/how-to-scale-a-tikzpicture-to-textwidth
\usepackage{environ}
\makeatletter
\newsavebox{\measure@tikzpicture}
\NewEnviron{scaletikzpicturetowidth}[1]{%
  \def\tikz@width{#1}%
  \def\tikzscale{1}\begin{lrbox}{\measure@tikzpicture}%
      \BODY
  \end{lrbox}%
  \pgfmathparse{#1/\wd\measure@tikzpicture}%
  \edef\tikzscale{\pgfmathresult}%
  \BODY
}
\makeatother


\usetikzlibrary{arrows,intersections}
\usetikzlibrary{shapes.misc}
\tikzset{cross/.style={cross out, draw=black, minimum size=2*(#1-\pgflinewidth), inner sep=0pt, outer sep=0pt},cross/.default={1pt}}



\makeatletter
\newcommand{\xleftrightarrow}[2][]{\ext@arrow 3359\leftrightarrowfill@{#1}{#2}}
\newcommand{\xdashrightarrow}[2][]{\ext@arrow 0359\rightarrowfill@@{#1}{#2}}
\newcommand{\xdashleftarrow}[2][]{\ext@arrow 3095\leftarrowfill@@{#1}{#2}}
\newcommand{\xdashleftrightarrow}[2][]{\ext@arrow 3359\leftrightarrowfill@@{#1}{#2}}
\def\rightarrowfill@@{\arrowfill@@\relax\relbar\rightarrow}
\def\leftarrowfill@@{\arrowfill@@\leftarrow\relbar\relax}
\def\leftrightarrowfill@@{\arrowfill@@\leftarrow\relbar\rightarrow}
\def\arrowfill@@#1#2#3#4{%
  $\m@th\thickmuskip0mu\medmuskip\thickmuskip\thinmuskip\thickmuskip
   \relax#4#1
   \xleaders\hbox{$#4#2$}\hfill
   #3$%
}

% alternative to arrow
\pgfarrowsdeclare{tonew}{tonew}
{
  \ifdim\pgfgetarrowoptions{tonew}=-1pt%
    \pgfutil@tempdima=0.84pt%
    \advance\pgfutil@tempdima by1.3\pgflinewidth%
    \pgfutil@tempdimb=0.21pt%
    \advance\pgfutil@tempdimb by.625\pgflinewidth%
  \else%
    \pgfutil@tempdima=\pgfgetarrowoptions{tonew}%
    \pgfarrowsleftextend{+-0.8\pgfutil@tempdima}%
    \pgfarrowsrightextend{+0.2\pgfutil@tempdima}%
  \fi%
}
{
  \ifdim\pgfgetarrowoptions{tonew}=-1pt%
    \pgfutil@tempdima=0.28pt%
    \advance\pgfutil@tempdima by.3\pgflinewidth%
    \pgfutil@tempdimb=0pt,%
  \else%
    \pgfutil@tempdima=\pgfgetarrowoptions{tonew}%
    \multiply\pgfutil@tempdima by 100%
    \divide\pgfutil@tempdima by 375%
    \pgfutil@tempdimb=0.4\pgflinewidth%
  \fi%
  \pgfsetdash{}{+0pt}
  \pgfsetroundcap
  \pgfsetroundjoin
  \pgfpathmoveto{\pgfpointorigin}
  \pgflineto{\pgfpointadd{\pgfpoint{0.75\pgfutil@tempdima}{0bp}}
                         {\pgfqpoint{-2\pgfutil@tempdimb}{0bp}}}
  \pgfusepathqstroke
  \pgfsetlinewidth{0.8\pgflinewidth}
  \pgfpathmoveto{\pgfpointadd{\pgfqpoint{-3\pgfutil@tempdima}{4\pgfutil@tempdima}}
                             {\pgfqpoint{\pgfutil@tempdimb}{0bp}}}
  \pgfpathcurveto
  {\pgfpointadd{\pgfqpoint{-2.75\pgfutil@tempdima}{2.5\pgfutil@tempdima}}
               {\pgfqpoint{0.5\pgfutil@tempdimb}{0bp}}}
  {\pgfpointadd{\pgfqpoint{0pt}{0.25\pgfutil@tempdima}}
               {\pgfqpoint{-0.5\pgfutil@tempdimb}{0bp}}}
  {\pgfpointadd{\pgfqpoint{0.75\pgfutil@tempdima}{0pt}}
               {\pgfqpoint{-\pgfutil@tempdimb}{0bp}}}
  \pgfpathcurveto
  {\pgfpointadd{\pgfqpoint{0pt}{-0.25\pgfutil@tempdima}}
               {\pgfqpoint{-0.5\pgfutil@tempdimb}{0bp}}}
  {\pgfpointadd{\pgfqpoint{-2.75\pgfutil@tempdima}{-2.5\pgfutil@tempdima}}
               {\pgfqpoint{0.5\pgfutil@tempdimb}{0bp}}}
  {\pgfpointadd{\pgfqpoint{-3\pgfutil@tempdima}{-4\pgfutil@tempdima}}
               {\pgfqpoint{\pgfutil@tempdimb}{0bp}}}
  \pgfusepathqstroke
}

% alias alternative to arrow
\pgfarrowsdeclarealias{<new}{>new}{tonew}{tonew}
\pgfsetarrowoptions{tonew}{-1pt}
\pgfkeys{/tikz/.cd, arrowhead/.default=-1pt, arrowhead/.code={
  \pgfsetarrowoptions{tonew}{#1},
}}
\makeatother


% \usepackage{url}
% \usepackage{xargs}
% \usepackage[pdftex,dvipsnames]{xcolor}
% \usepackage[colorinlistoftodos,prependcaption,textsize=tiny]{todonotes}
% \usepackage[small,bf]{caption}


% \usepackage{hyperref}
% \hypersetup{
    % colorlinks,
    % citecolor=red,
    % filecolor=red,
    % linkcolor=red,
    % urlcolor=red
% }


% \usepackage{fancybox}
% \usepackage{multirow}
% \usepackage{nicefrac}
% \usepackage{hyperref}


% \expandafter\def\expandafter\normalsize\expandafter{%
    % \normalsize
% %	    \setlength\abovedisplayskip{5pt}
    % \setlength\belowdisplayskip{5pt}
    % % \setlength\abovedisplayshortskip{5pt}
    % \setlength\belowdisplayshortskip{5pt}
% }

% \usepackage{tocloft}
% \setlength{\cftsecnumwidth}{2.8em}%



\makeatother

\usepackage{babel}
\begin{document}

\section{Philosophy}

Often, optimization methods follow the following framework:

\begin{algorithm2e}[H]

\caption{$\mathtt{OptimizationFramework}$}

\SetAlgoLined

\For{$k=0,1,\cdots$}{

Approximate $f$ by a simpler function $f_{k}$ according to the current
point $x^{(k)}$

Do something using $f_{k}$ (such as set $x^{(k+1)}=\arg\min_{x}f_{k}(x)$)

}

\end{algorithm2e}

The runtime depends on the number of iterations and the cost per iteration.
Philosophically, difficulties of a problem can never be created nor
destroyed, only converted from one form of difficulty to another.
When we decrease the number of iterations, the cost per iteration
often increase. The gain of new methods often come from avoiding some
wasted computation, utilizing some forgotten information or giving
a faster but tailored algorithm for a sub-problem.

One key question to answer in designing an optimization algorithm
is that what the problem looks like (or how can we approximate $f$
by a simpler function). Here are some approximation we will used in
this textbook:
\begin{itemize}
\item First-order Approximation: $f(y)\approx f(x)+\left\langle \nabla f(x),y-x\right\rangle $
(Section \ref{sec:Gradient-Descent})
\item Second-order Approximation: $f(y)\approx f(x)+\left\langle \nabla f(x),y-x\right\rangle +(y-x)^{\top}\nabla^{2}f(x)(y-x)$
(Section \ref{sec:Newton-Method})
\item Stochastic Approximation: $\sum_{i}f_{i}(x)\approx f_{j}(x)$ for
a random $j$ (Section \ref{sec:Stochastic-Gradient-Descent})
\item Matrix Approximation: Approximate $A$ by a simpler $B$ with $\frac{1}{2}A\preceq B\preceq2A$
(Section \ref{sec:Subspace-embedding} and Section \ref{sec:Leverage-Score-Sampling})
\item Set Approximation: Approximate a convex set by an ellipsoid or a polytope
(Section \ref{sec:Cutting-Plane-Methods})
\item Barrier Approximation: Approximate a convex set by a smooth function
that blows up on the boundary (Section \ref{sec:IPM})
\item Polynomial Approximation: Approximate a function by a polynomial (Section
\ref{sec:Chebyshev-Polynomials})
\item Partial Approximation: Split the problem into two parts and approximate
only one part
\end{itemize}
Here are other approximation not covered:
\begin{itemize}
\item Taylor Approximation: $f(y)\approx\sum_{k=0}^{K}D^{k}f(x)[y-x]^{k}$
\item Mixed $\ell^{2}$-$\ell^{p}$ Approximation: $f(y)\approx f(x)+\left\langle \nabla f(x),y-x\right\rangle +\sum_{i=1}^{n}\alpha_{i}(y_{i}-x_{i})^{2}+\beta_{i}(y_{i}-x_{i})^{p}$
\item Stochastic Matrix Approximation: Approximate $A$ by a simpler random
$B$ with $B\preceq2A$ and $\E B\succeq\frac{1}{2}A$ 
\item Homotopy Method: Approximate a function by a family of functions 
\item ...(Please give me more examples here)...
\end{itemize}
The second question to answer is how we maintain all the different
approximations we created in each step. One simple way would be forget
the approximation we got in previous steps, but this is often not
optimal. Another way is to keep all previous approximations/information
(such as Section \ref{sec:Cutting-Plane-Methods}). Often the best
way will be combining previous and current approximation carefully
to a better approximation (such as Section \ref{sec:Accelerated-Gradient-Descent}).

\section{Basic Algorithm\label{sec:Gradient-Descent}}

Perhaps the most natural algorithm for optimization is gradient descent.
In fact it has many variants with different guarantees. Assume that
the function $f$ to be optimized is continuously differentiable.
By basic calculus, either the minimum (or point achieving the minimum)
is unbounded or the gradient is zero at a minimum. So we try to find
a point with gradient close to zero (which, of course, does not guarantee
global optimality). The basic algorithm is the following:

\begin{algorithm2e}[H]

\caption{$\mathtt{GradientDescent}$ (GD)}

\SetAlgoLined

\textbf{Input:} Initial point $x^{(0)}\in\Rn$, step size $h>0$.

\For{$k=0,1,\cdots$}{

\lIf{$\|\nabla f(x^{(k)})\|_{2}\leq\epsilon$}{\Return$x^{(k)}$}

\tcp{Alternatively, one can use $x^{(k+1)}\leftarrow\text{argmin}_{x=x^{(k)}+t\nabla f(x^{(k)})}f(x)$.}

$x^{(k+1)}\leftarrow x^{(k)}-h\cdot\nabla f(x^{(k)})$.

}

\end{algorithm2e}

One can view gradient descent as a greedy method for solving $\min_{x\in\Rn}f(x)$.
At a point $x$, gradient descent goes to the minimizer of 
\[
\min_{\|\Delta\|_{2}\leq\eta}f(x)+\nabla f(x)^{\top}\Delta.
\]
The term $f(x)+\nabla f(x)^{\top}\Delta$ is simply the first-order
approximation of $f(x+\Delta)$. Note that in this problem, the current
point $x$ is fixed and we are optimizing the step $\Delta$. Certainly,
there is no inherent reason for using first-order approximation and
the Euclidean norm $\|x\|_{2}$. For example, if you use second-order
approximation, then you would get a method involving Hessian of $f$.

The step size of the algorithm usually either uses a fixed constant,
or follows a predetermined schedule, or determined using a line search.

If the iteration stops, we get a point with $\norm{\nabla f(x)}_{2}\le\epsilon$.
Why is this good? The hope is that $x$ is a near-minimum in the neighborhood
of $x$. However, this might not be true if the gradient can fluctuate
wildly:
\begin{defn}
We call $f$ has $L$-Lipschitz gradient if $\nabla f$ is $L$-Lipschitz,
namely, $\|\nabla f(x)-\nabla f(y)\|_{2}\leq L\|x-y\|_{2}$ for all
$x,y$.

Similar to Theorem \ref{thm:equ_convex}, we have the following equivalent:
\end{defn}

\begin{thm}
\label{thm:L_equ_def}Let $f\in\mathcal{C}^{2}(\Rn)$. For any $\mu\geq0$,
the following are equivalent:
\begin{enumerate}
\item $\|\nabla f(x)-\nabla f(y)\|_{2}\leq L\|x-y\|_{2}$ for all $x,y\in\Rn$.
\item $-L\preceq\nabla^{2}f(x)\preceq L$ for all $x\in\Rn$.
\item $f(y)=f(x)+\nabla f(x)^{\top}(y-x)\pm\frac{L}{2}\|y-x\|_{2}^{2}$
for all $x,y\in\Rn$.
\end{enumerate}
\end{thm}

\begin{proof}
Suppose (1) holds. By the definition of $\nabla^{2}f$, we have
\[
\nabla^{2}f(x)v=\lim_{h\rightarrow0}\frac{\nabla f(x+hv)-\nabla f(x)}{h}.
\]
Since $\|\frac{\nabla f(x+hv)-\nabla f(x)}{h}\|_{2}\leq\frac{L}{h}\|hv\|_{2}=L\|v\|_{2}$,
we have $\|\nabla^{2}f(x)v\|_{2}\leq L\|v\|_{2}$. This proves (2).

Suppose (2) holds. Since $\nabla f(x)-\nabla f(y)=\int_{0}^{1}\nabla^{2}f(y+t(x-y))(x-y)dt$,
we have that
\[
\|\nabla f(x)-\nabla f(y)\|_{2}\leq\int_{0}^{1}\|\nabla^{2}f(y+t(x-y))\|_{\op}\|x-y\|_{2}dt\leq L\|x-y\|_{2}.
\]
This gives (1).

Suppose (2) holds. By Taylor expansion, we have
\[
f(y)=f(x)+\nabla f(x)^{\top}(y-x)+\int_{0}^{1}(1-t)(y-x)^{\top}\nabla^{2}f(x+t(y-x))(y-x)dt.
\]
Since $-L\preceq\nabla^{2}f(x)\preceq L$, we have
\[
(y-x)^{\top}\nabla^{2}f(x+t(y-x))(y-x)=\pm L\|y-x\|^{2}.
\]
Using this gives (3).

Suppose (3) holds, then $g(x)=f(x)+\frac{L}{2}\|x\|^{2}$ satisfies
$g(y)\geq g(x)+\nabla g(x)^{\top}(y-x)$ for all $x,y\in\Rn$. Theorem
\ref{thm:equ_convex} shows that $g$ is convex and $\nabla^{2}g(x)\succeq0$.
This shows that $\nabla^{2}f(x)\succeq-L$. Similarly, by taking $g(x)=\frac{L}{2}\|x\|^{2}-f(x)$,
we have $\nabla^{2}f(x)\preceq L$. Hence, this gives (2).
\end{proof}
With this equivalent definition, we can have an alternative view of
gradient descent. Each step, we perform
\[
x^{(k+1)}=\arg\min_{y}f(x^{(k)})+\left\langle \nabla f(x^{(k)}),y-x^{(k)}\right\rangle +\frac{L}{2}\|y-x^{(k)}\|_{2}^{2}.
\]
To see this is the same step, we let $g(y)=f(x^{(k)})+\left\langle \nabla f(x^{(k)}),y-x^{(k)}\right\rangle +\frac{L}{2}\|y-x^{(k)}\|_{2}^{2}$.
The optimality condition shows that $0=\nabla g(x^{(k+1)})=\nabla f(x^{(k)})+L(x^{(k+1)}-x^{(k)})$.
Hence, this gives the step $x^{(k+1)}=x^{(k)}-\frac{1}{L}\nabla f(x^{(k)})$.

By Theorem \ref{thm:L_equ_def}, we know that $g$ is an upper bound
of $f$, namely $g(y)\geq f(y)$ for all $y$. In general, many optimization
methods involves minimizing some upper bound function every step.
Note that the progress we made for $f$ is at least the progress we
made for $g$. If $g$ is exactly $f$, we can get all the progress
we can make in one step. Hence, we should believe if $g$ is a better
approximation of $f$, then we are making more progress. For gradient
descent, it uses the simplest first-order approximation. Although
this is not the best approximation one can come up, but it is robust
enough to use in all sort of applications.

\subsection{Analysis for general functions}

Gradient descent works for both convex and non-convex functions. For
non-convex function, we can only find a point with small gradient
(called an approximate saddle point).
\begin{thm}
\label{thm:gd_general}Let $f$ be a function with $L$-Lipschitz
gradient and $x^{*}$ be any minimizer of $f$. The $\mathtt{GradientDescent}$
with step size $h=\frac{1}{L}$ outputs a point $x$ such that $\|\nabla f(x)\|_{2}\leq\epsilon$
in $\frac{2L}{\epsilon^{2}}\left(f(x^{(0)})-f(x^{*})\right)$ iterations.
\end{thm}

The proof idea involves showing the function value $f(x)$ decreases
by at least $\frac{\epsilon^{2}}{2L}$ when $\|\nabla f(x)\|_{2}\geq\epsilon$.
Since the function value can only decrease by at most $f(x^{(0)})-f(x^{*})$,
this bounds the number of iterations.
\begin{lem}
\label{lem:gradient_progress}For any $f\in\mathcal{C}^{2}(\R^{n})$
with $L$-Lipschitz gradient, we have
\[
f(x-\frac{1}{L}\nabla f(x))\leq f(x)-\frac{1}{2L}\|\nabla f(x)\|_{2}^{2}.
\]
\end{lem}

\begin{proof}
Lemma \ref{lem:second_order} shows that 
\[
f(x-\frac{1}{L}\nabla f(x))=f(x)-\frac{1}{L}\|\nabla f(x)\|_{2}^{2}+\frac{1}{2L^{2}}\nabla f(x)^{\top}\nabla^{2}f(z)\nabla f(x)
\]
for some $z\in[x,x-\frac{1}{L}\nabla f(x)]$. Since $\|\nabla^{2}f(x)\|_{\op}\leq L$,
we have that $\nabla f(x)^{\top}\nabla^{2}f(z)\nabla f(x)\leq L\cdot\|\nabla f(x)\|_{2}^{2}$.
Hence, we have the result.
\end{proof}
%
\begin{proof}[Proof of Theorem \ref{thm:gd_general}]
Since each step of gradient descent decreases $f$ by at least $\frac{\epsilon^{2}}{2L}$
and since we can decrease $f$ by at most $f(x^{(0)})-f^{*}$, we
have the result.
\end{proof}
Despite the simplicity of the algorithm and the proof, it is known
that this is the best one can do via any algorithm for this general
setting \cite{carmon2017lower}.

\section{Analysis for convex functions}

Assuming the function is convex, we can prove that gradient descent
in fact converges to the global minimum. In particular, when $\|\nabla f(x)\|_{2}$
is small, convexity shows that $f(x)-f^{*}$ is small (Theorem \ref{thm:first_order}).
\begin{lem}
\label{lem:f_diff_diameter}For any convex $f\in\mathcal{C}^{1}(\Rn)$,
we have that $f(x)-f(y)\leq\|\nabla f(x)\|_{2}\cdot\|x-y\|_{2}$ for
all $x,y$.
\end{lem}

\begin{proof}
Theorem \ref{thm:first_order} and Cauchy-Schwarz inequality shows
that 
\[
f(x)-f(y)\leq\left\langle \nabla f(x),x-y\right\rangle \leq\|\nabla f(x)\|_{2}\cdot\|x-y\|_{2}.
\]
\end{proof}
This in turns give a better bound on the number of iterations because
the bound in Theorem \ref{thm:gd_general} is affected by $f(x)-f^{*}$.
\begin{thm}
\label{thm:gd_convex}Let $f\in\mathcal{C}^{2}(\R^{n})$ be convex
with $L$-Lipschitz gradient and $x^{*}$ be any minimizer of $f$.
With step size $h=\frac{1}{L}$, the sequence $x^{(k)}$ in $\mathtt{GradientDescent}$
satisfies
\[
f(x^{(k)})-f(x^{*})\leq\frac{2LR^{2}}{k+4}\text{ where }R=\max_{f(x)\leq f(x^{(0)})}\|x-x^{*}\|_{2}.
\]
\end{thm}

\begin{proof}
Let $\epsilon_{k}=f(x^{(k)})-f(x^{*})$. Lemma \ref{lem:gradient_progress}
shows that
\[
f(x^{(k+1)})=f(x^{(k)}-\frac{1}{L}\nabla f(x^{(k)}))\leq f(x^{(k)})-\frac{1}{2L}\|\nabla f(x^{(k)})\|_{2}^{2}.
\]
Subtracting $f(x^{*})$ from both sides, we have $\epsilon_{k+1}\leq\epsilon_{k}-\frac{1}{2L}\|\nabla f(x^{(k)})\|_{2}^{2}$.
Lemma \ref{lem:f_diff_diameter} shows that 
\[
\epsilon_{k}\leq\|\nabla f(x^{(k)})\|_{2}\cdot\|x^{(k)}-x^{*}\|_{2}\leq\|\nabla f(x^{(k)})\|_{2}\cdot R.
\]
Therefore, we have that
\[
\epsilon_{k+1}\leq\epsilon_{k}-\frac{1}{2L}\left(\frac{\epsilon_{k}}{R}\right)^{2}.
\]
Also, we have that 
\[
\epsilon_{0}=f(x^{(0)})-f^{*}\leq\nabla f(x^{*})^{\top}(x^{(0)}-x^{*})+\frac{L}{2}\|x^{(0)}-x^{*}\|_{2}^{2}\leq\frac{LR^{2}}{2}.
\]
Now, we need to solve the recursion. We note that
\[
\frac{1}{\epsilon_{k+1}}-\frac{1}{\epsilon_{k}}=\frac{\epsilon_{k}-\epsilon_{k+1}}{\epsilon_{k}\epsilon_{k+1}}\geq\frac{\epsilon_{k}-\epsilon_{k+1}}{\epsilon_{k}^{2}}\geq\frac{1}{2LR^{2}}.
\]
Therefore, after $k$ iterations, we have
\[
\frac{1}{\epsilon_{k}}\geq\frac{1}{\epsilon_{0}}+\frac{k}{2LR^{2}}\geq\frac{2}{LR^{2}}+\frac{k}{2LR^{2}}=\frac{k+4}{2LR^{2}}.
\]
\end{proof}
This style of proof is typical in optimization. It shows that when
the gradient is large, then we make large progress and when the gradient
is small, we are close to optimal. 

This proof did not use any property of $\ell_{2}$ or inner product
space. Therefore, it works for general norms if the gradient descent
step is defined using that norm. For the case of $\ell_{2}$, one
can prove that $\|x^{(k)}-x^{*}\|_{2}$ is in fact decreasing:
\begin{lem}
For $h\leq\frac{2}{L}$, we have that $\|x^{(k+1)}-x^{*}\|_{2}\leq\|x^{(k)}-x^{*}\|_{2}$.
Therefore, for $h=\frac{1}{L}$, we have
\[
f(x^{(k)})-f(x^{*})\leq\frac{2L\|x^{(0)}-x^{*}\|_{2}^{2}}{k+4}.
\]
\end{lem}

\begin{proof}
We compute the distance as follows:
\begin{align*}
\|x^{(k+1)}-x^{*}\|_{2}^{2} & =\|x^{(k)}-x^{*}-h\nabla f(x^{(k)})\|_{2}^{2}\\
 & =\|x^{(k)}-x^{*}\|_{2}^{2}-2h\left\langle \nabla f(x^{(k)}),x^{(k)}-x^{*}\right\rangle +h^{2}\|\nabla f(x^{(k)})\|^{2}\\
 & =\|x^{(k)}-x^{*}\|_{2}^{2}-2h\left\langle \nabla f(x^{(k)})-\nabla f(x^{*}),x^{(k)}-x^{*}\right\rangle +h^{2}\|\nabla f(x^{(k)})-\nabla f(x^{*})\|^{2}.
\end{align*}
To handle the term $\nabla f(x)-\nabla f(x^{*})$, we note that
\[
\nabla f(x^{(k)})-\nabla f(x^{*})=H(x^{(k)}-x^{*})
\]
with $H=\int_{0}^{1}\nabla^{2}f(x^{*}+t(x^{(k)}-x^{*}))dt$. Since
$0\preceq H\preceq L$ and that $H\succeq\frac{1}{L}H^{2}$, we have
\begin{align*}
\left\langle \nabla f(x^{(k)})-\nabla f(x^{*}),x^{(k)}-x^{*}\right\rangle  & =(x^{(k)}-x^{*})^{\top}H(x^{(k)}-x^{*})\\
 & \geq\frac{1}{L}(x^{(k)}-x^{*})^{\top}H^{2}(x^{(k)}-x^{*})\\
 & =\frac{1}{L}\|\nabla f(x^{(k)})-\nabla f(x^{*})\|^{2}.
\end{align*}
Hence, we have
\begin{align*}
\|x^{(k+1)}-x^{*}\|_{2}^{2} & \leq\|x^{(k)}-x^{*}\|_{2}^{2}-(\frac{2h}{L}-h^{2})\|\nabla f(x^{(k)})-\nabla f(x^{*})\|^{2}\\
 & \leq\|x^{(k)}-x^{*}\|_{2}^{2}.
\end{align*}
The error estimate follows from $\|x^{(k)}-x^{*}\|_{2}^{2}\leq\|x^{(0)}-x^{*}\|_{2}^{2}$
for all $k$ and the proof in Theorem \ref{thm:gd_convex}.
\end{proof}
Rewriting the bound, Theorem \ref{thm:gd_convex} shows it takes $\frac{2L\|x^{(0)}-x^{*}\|_{2}^{2}}{\epsilon}$
iterations. Compare to the bound $\frac{2L}{\epsilon^{2}}\left(f(x^{(0)})-f^{*}\right)$
in Theorem \ref{thm:gd_general}, it seems the new result has a strictly
better dependence on $\epsilon$. However, this is not true because
one is measuring the error in terms of $\|\nabla f(x)\|_{2}$ and
one is measuring the error in terms of $f(x)-f^{*}$. For $f(x)=x^{2}/2$,
we have $f(x)-f^{*}=\|\nabla f(x)\|_{2}^{2}$ and hence both have
the same dependence on $\epsilon$ for this particular function. So,
the real benefit of Theorem \ref{thm:gd_convex} is its global convergence.

\section{Strongly Convex Functions}

We note that the convergence rate $\epsilon^{-1}$ or $\epsilon^{-2}$
is not great if we need to solve the problem up to machine accuracy.
Getting to machine accuracy is sometimes important if the optimization
problem is used as a subroutine. We note that for the case $f(x)=\frac{1}{2}\|x\|_{2}^{2}$,
gradient descent with step size $h=1$ takes exactly $1$ step. Therefore,
it is natural to ask if one can improve the bound for functions close
to quadratics. This motivates the following assumption:
\begin{defn}
We call a function $f\in\mathcal{C}^{1}(\Rn)$ is $\mu$-strongly
convex if for any $x,y\in\Rn$ 
\[
f(y)\geq f(x)+\nabla f(x)^{\top}(y-x)+\frac{\mu}{2}\|y-x\|_{2}^{2}.
\]
\end{defn}

Similar to the convex case (Theorem \ref{thm:equ_convex}), we have
the following:
\begin{thm}
Let $f\in\mathcal{C}^{2}(\Rn)$. For any $\mu\geq0$, the following
are equivalent:
\begin{itemize}
\item $f(\lambda x+(1-\lambda)y)\leq\lambda f(x)+(1-\lambda)f(y)-\frac{1}{2}\mu\lambda(1-\lambda)\|x-y\|_{2}^{2}$
for all $x,y\in\Rn$ and $\lambda\in[0,1]$.
\item $f(y)\geq f(x)+\nabla f(x)^{\top}(y-x)+\frac{\mu}{2}\|y-x\|_{2}^{2}$
for all $x,y\in\Rn$.
\item $\nabla^{2}f(x)\succeq\mu$ for all $x\in\Rn$
\end{itemize}
\end{thm}

\begin{proof}
It follows from applying Theorem \ref{thm:equ_convex} on the function
$g(x)=f(x)-\frac{\mu}{2}\|x\|^{2}$.
\end{proof}
Now, we study gradient descent for $\mu$-strongly convex functions.
\begin{thm}
\label{thm:gd_convex_strongly}Let $f\in\mathcal{C}^{2}(\R^{n})$
be $\mu$-strongly convex with $L$-Lipschitz gradient and $x^{*}$
be any minimizer of $f$. With step size $h=\frac{1}{L}$, the sequence
$x^{(k)}$ in $\mathtt{GradientDescent}$ satisfies
\[
f(x^{(k)})-f(x^{*})\leq(1-\frac{\mu}{L})^{k}(f(x^{(0)})-f(x^{*})).
\]
\end{thm}

In a later chapter, we will see that an \emph{accelerated }variant
of gradient descent improves this further by replacing the $\frac{\mu}{L}$
term with $\sqrt{\frac{\mu}{L}}$.
\begin{proof}
Lemma \ref{lem:gradient_progress} shows that
\begin{equation}
f(x^{(k+1)})-f^{*}\leq f(x^{(k)})-f^{*}-\frac{1}{2L}\|\nabla f(x^{(k)})\|_{2}^{2}.\label{eq:strong_convex_progress}
\end{equation}
Next, the definition of $\mu$ strong convexity implies that
\[
f(x^{*})\geq f(x^{(k)})+\nabla f(x^{(k)})^{\top}(x^{*}-x^{(k)})+\frac{\mu}{2}\|x^{*}-x^{(k)}\|_{2}^{2}.
\]
Rearranging the term, we have
\begin{equation}
f(x^{(k)})-f^{*}\leq\nabla f(x^{(k)})^{\top}(x^{(k)}-x^{*})-\frac{\mu}{2}\|x^{(k)}-x^{*}\|_{2}^{2}\leq\max_{\Delta}\left(\nabla f(x^{(k)})^{\top}\Delta-\frac{\mu}{2}\|\Delta\|_{2}^{2}\right)=\frac{1}{2\mu}\|\nabla f(x^{(k)})\|_{2}^{2}.\label{eq:err_vs_grad}
\end{equation}
Putting this into the gradient term in (\ref{eq:strong_convex_progress})
gives
\[
f(x^{(k+1)})-f^{*}\leq(1-\frac{\mu}{L})(f(x^{(k)})-f^{*}).
\]

The conclusion follows.
\end{proof}
It is natural to ask to what extent the assumption of convexity is
essential for the bounds we obtained. This is the motivation for the
next exercises.
\begin{xca}
Suppose $f$ satisfies $\left\langle \nabla f(x),x-x^{*}\right\rangle \ge\alpha\left(f(x)-f(x^{*})\right)$.
Derive a bound similar to Theorem \ref{thm:gd_convex} for gradient
descent.
\end{xca}

\begin{xca}
Suppose $f$ satisfies $\norm{\nabla f(x)}_{2}^{2}\ge\mu\left(f(x)-f(x^{*})\right).$
Derive a bound similar to Theorem \ref{thm:gd_convex_strongly} for
gradient descent.
\end{xca}

\begin{xca}
Give examples of nonconvex functions satisfying the above conditions.
(Note: convex functions satisfy the first with $\alpha=1$ and $\mu$-strongly
convex functions satisfy the second.)
\end{xca}


\section{Line Search}

In practice, we often stop the line search early because we can make
larger progress via a new direction. One standard stopping condition
is called Wolfe conditions.
\begin{defn}
A step size $h$ satisfies the Wolfe conditions with respect to direction
$p$ if
\begin{enumerate}
\item $f(x+hp)\leq f(x)+c_{1}h\cdot p^{\top}\nabla f(x)$,
\item $-p^{\top}\nabla f(x+hp)\leq-c_{2}p^{\top}\nabla f(x)$.
\end{enumerate}
with $0<c_{1}<c_{2}<1$.
\end{defn}

Let the object progress $\phi(h)=f(x)-f(x+hp)$. The first condition
requires the algorithm makes sufficient progress ($\phi(h)\geq c_{1}h\cdot\phi'(0)$).
The second condition requires the slop reduced significantly ($\phi'(h)\leq c_{2}\phi'(0)$).
One can think the first condition gives a upper bound on $h$ while
the second condition gives a lower bound on $h$. In general, step
size satisfying the Wolfe conditions will be larger than the step
size $h=\frac{1}{L}$. In particular, one can show the following:
\begin{xca}
Suppose $f$ is $\mu$-strongly convex and has $L$-Lipschitz gradient.
If a step size $h$ satisfies the Wolfe conditions with the direction
$p=-\nabla f(x)$, then we have
\[
\frac{2(1-c_{1})}{\mu}\geq h\geq\frac{1-c_{2}}{L}.
\]
\end{xca}

As a corollary, we have that the function value progress given by
such step is at least $\Omega(\|\nabla f(x)\|^{2}/L)$. Therefore,
this gives the same guarantee Theorem \ref{thm:gd_convex} and Theorem
\ref{thm:gd_convex_strongly}. One common way to via backtracking
line search. The algorithm starts with a large step size and decreases
it by a constant factor when the Wolfe conditions are violated. For
gradient descent, the next step involves exactly computing $\nabla f(x+hp)$
and hence if our line search accepts the step size immediately, the
line search almost cost nothing. Therefore, if we maintain a step
size throughout the algorithm and decreases it only when it violates
the condition, the total cost of the line search will be only an additive
logarithmic number of gradient calls throughout the algorithm and
it is negligible.

Finally, we note that for problems of the form $\sum_{i}f_{i}(a_{i}^{\top}x)$,
the bottleneck is often in computing $Ax$. In this case, exact line
search is almost free because we can store the vectors $Ax$ and $Ah$.

\section{Generalizing Gradient Descent}

Now, we study what properties gradient descent are using for the strongly
convex case. There are many ways to generalize it. One way is to view
gradient descent approximate the function $f$ by splitting it into
two terms, one term is the first-order approximation and the second
term is just an $\ell_{2}$ norm. More generally, we can split a function
into two terms, one term is easy to optimize and the another term
we need to approximate with some error. Precisely, we consider the
following 
\begin{defn}
We say $g+h$ is a $\alpha$-approximation to $f$ at $x$ if
\begin{itemize}
\item $g$ is convex, $g(x)=f(x)$,
\item $h(x)=0$ and $h((1-\alpha)x+\lambda\widehat{x})\leq\lambda^{2}h(\widehat{x})$
for all $\widehat{x}$,
\item $g(y)+\alpha h(y)\leq f(y)\leq g(y)+h(y)\text{ for all }y.$ 
\end{itemize}
\end{defn}

To understand this assumption, we note that if $f$ is $\mu$-strongly
convex with $L$-Lipschitz gradient, then for any $x$, we can use
$\alpha=\frac{\mu}{L}$ and
\begin{align*}
g(y) & =f(x)+\left\langle \nabla f(x),y-x\right\rangle ,\\
h(y) & =\frac{L}{2}\|y-x\|^{2}.
\end{align*}
The condition requires $h$ converging to $0$ quadratically when
$y\rightarrow x$. 

Now, we consider the following algorithm:

\begin{algorithm2e}[H]

\caption{$\mathtt{GeneralizedGradientDescent}$}

\SetAlgoLined

\textbf{Input:} Initial point $x^{(0)}\in\Rn$, approximation factor
$\alpha>0$.

\For{$k=0,1,\cdots$}{

Find a $\alpha$-approximation of $f$ at $x^{(k)}$ given by $g^{(k)}(x)+h^{(k)}(x)$.

$x^{(k+1)}\leftarrow\arg\min_{y}g^{(k)}(y)+h^{(k)}(y)$.

}

\end{algorithm2e}
\begin{thm}
\label{thm:gd_general_apx}Given a convex function $f$ that we can
find a $\alpha$-approximation at any $x$. Let $x^{*}$ be any minimizer
of $f$. Then the sequence $x^{(k)}$ in $\mathtt{GeneralizedGradientDescent}$
satisfies
\[
f(x^{(k)})-f(x^{*})\leq(1-\alpha)^{k}(f(x^{(0)})-f(x^{*})).
\]
\end{thm}

\begin{proof}
Using that $g^{(k)}+h^{(k)}$ is an upper bound of $f$, we have that
our progress on $f$ is larger than the best possible progress on
$g^{(k)}+h^{(k)}$:
\[
f(x^{(k+1)})\leq\min_{y}g^{(k)}(y)+h^{(k)}(y).
\]
To bound the best possible progress, we consider $\widehat{x}=\arg\min_{y}g^{(k)}(y)+\alpha h^{(k)}(y)$
and $z=(1-\alpha)x^{(k)}+\alpha\widehat{x}$. We have that
\begin{align*}
\min_{y}g^{(k)}(y)+h^{(k)}(y) & \leq g^{(k)}(z)+h^{(k)}(z)\\
 & \leq(1-\alpha)g^{(k)}(x^{(k)})+\alpha g^{(k)}(\widehat{x})+\alpha^{2}h^{(k)}(\widehat{x})\\
 & \leq(1-\alpha)g^{(k)}(x^{(k)})+\alpha(g^{(k)}(x^{*})+\alpha h^{(k)}(x^{*}))
\end{align*}
where we used $g^{(k)}$ is convex and the assumption on $h$ in the
second inequality, we used $\widehat{x}$ minimizes $g^{(k)}+\alpha h^{(k)}$.

Combining both and using $g^{(k)}+\alpha h^{(k)}$ is a lower bound
of $f$, we have
\[
f(x^{(k+1)})\leq(1-\alpha)f(x^{(k)})+\alpha f(x^{*}).
\]
This gives the result.
\end{proof}
Although Theorem \ref{thm:gd_general_apx} looks weird, it captures
many theorems. Here we list some of them.

\subsubsection*{Projected Gradient Descent / Proximal Gradient Descent}

Given a convex set $K$, a $\mu$-strongly convex function $f$ with
$L$-Lipschitz gradient, we consider the problem
\[
\min_{x\in K}f(x).
\]
To apply Theorem \ref{thm:gd_general_apx} to $F(x)\defeq f(x)+\delta_{K}(x)$,
for any $x$, we consider the functions
\begin{align*}
g(y) & =f(x)+\left\langle \nabla f(x),y-x\right\rangle +\delta_{K}(y),\\
h(y) & =\frac{L}{2}\|y-x\|_{2}^{2}.
\end{align*}
Note that $g+h$ is a $\frac{\mu}{L}$-approximation to $F$. Theorem
\ref{thm:gd_general_apx} shows the $\mathtt{GeneralizedGradientDescent}$
converges in $O(\frac{L}{\mu}\log(1/\epsilon))$ steps where each
step involves solving the problem
\[
\min_{y\in K}f(x)+\left\langle \nabla f(x),y-x\right\rangle +\frac{L}{2}\|y-x\|_{2}^{2}.
\]

More generally, this works for problems of the form
\[
\min_{x}f(x)+\phi(x)
\]
for some convex function $\phi(x)$. Theorem \ref{thm:gd_general_apx}
requires us to solve a sub-problem of the form
\[
\min_{y}f(x)+\left\langle \nabla f(x),y-x\right\rangle +\frac{L}{2}\|y-x\|_{2}^{2}+\phi(y).
\]


\subsubsection*{$\ell^{p}$ Regression}

One can apply this framework for optimizing $\ell^{p}$ regression
$\min_{x}\|Ax-b\|_{p}^{p}$. For example, one can approximate the
function $f(x)=x^{p}$ by $g(y)=x^{p}+px^{p-1}(y-x)$ and $h(y)=p2^{p-1}(x^{p-2}(y-x)^{2}+(y-x)^{p})$.
Using this, one can show that one can solve the problem
\[
\min_{x}\|Ax-b\|_{p}^{p}
\]
using the problem
\[
\min_{y}v^{\top}y+\|DA(y-x)\|_{2}^{2}+\|A(y-x)\|_{p}^{p}
\]
for some vector $v$ and some diagonal matrix $D$. One can show that
Theorem \ref{thm:gd_general_apx} only need to solve sub-problem approximately.
Therefore, this shows that one can solve $\ell^{p}$ regression with
$\log(1/\epsilon)$ converging by solving mixed $\ell_{2}+\ell_{p}$
regression approximately. 

Recently, \cite{adil2019iterative} and \cite{kyng2019flows} applied
these to obtain the fastest algorithms for $\ell^{p}$ regression
and the $\ell^{p}$ flow problem. \cite{kathuria2020unit} showed
that the $\ell^{p}$ flow problem can be used as a subroutine to solve
uncapacitied maximum flow problem in $m^{4/3+o(1)}$ time.

\subsubsection*{Other assumptions}

Instead of assuming $h(x)$ converges to $0$ quadratically, \cite{lu2016relatively}
proved Theorem \ref{thm:gd_general_apx} assuming $h$ is given by
some divergence and showed its applications in D-optimal design.

For some special case, it is possible to analyze this algorithm without
convexity. One prominent application is compressive sensing: 
\[
\min_{\|x\|_{0}\leq k}\|Ax-b\|_{2}^{2}.
\]
For matrices $A$ satisfying restricted isometry property, one can
apply $\mathtt{GeneralizedGradientDescent}$ to solve the problem
with the splitting $g(x)=2A^{\top}(Ax-b)+\delta_{\|x\|_{0}\leq k}$
and $h(x)=\|x\|^{2}$. In this case, the algorithm is called iterative
hard-thresholding \cite{blumensath2009iterative} and the sub-problem
has a closed form expression.
\begin{xca}
Give the close form solution for the sub-problem given by the splitting
above.
\end{xca}


\section{Gradient Flow}

In continuous time, gradient descent follows the ODE 
\[
\frac{dx_{t}}{dt}=-\nabla f(x_{t}).
\]
This can be viewed as the canonical continuous algorithm. Finding
the right discretization has lead to many fruitful research directions.
One benefit of the continuous view is to simplify some calculations.
For example, the Theorem \ref{thm:gd_convex_strongly} now becomes
\[
\frac{d}{dt}(f(x_{t})-f(x^{*}))=\nabla f(x_{t})^{\top}\frac{dx_{t}}{dt}=-\|\nabla f(x_{t})\|_{2}^{2}\leq-2\mu(f(x_{t})-f(x^{*}))
\]
where we used (\ref{eq:err_vs_grad}) at the end. Solving this differential
inequality, we have
\[
f(x_{t})-f(x^{*})\leq e^{-2\mu t}(f(x_{0})-f(x^{*})).
\]
Without the strongly convexity assumption, the gradient flow can behave
wildly. For example, the length of the gradient flow can be exponential
in $d$ on an unit ball \cite{manselli1991maximum}. Finally, we emphasize
that this continuous view is mainly useful for understanding, indicative
of but not necessarily an algorithmic result.
\end{document}
