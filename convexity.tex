%% LyX 2.3.6.1 created this file.  For more info, see http://www.lyx.org/.
%% Do not edit unless you really know what you are doing.
\documentclass[oneside,english]{book}
\usepackage[T1]{fontenc}
\usepackage[latin9]{inputenc}
\setcounter{secnumdepth}{3}
\setcounter{tocdepth}{3}
\usepackage{amsmath}
\usepackage{amsthm}
\usepackage{amssymb}

\makeatletter
%%%%%%%%%%%%%%%%%%%%%%%%%%%%%% Textclass specific LaTeX commands.
\usepackage[margin=1in]{geometry}
\usepackage[lined,boxed,ruled,norelsize,algo2e]{algorithm2e}

\theoremstyle{plain}
\newtheorem{thm}{Theorem}[chapter]
\newtheorem*{thm*}{{Theorem}}
\newtheorem{cor}[thm]{Corollary}
\newtheorem*{cor*}{Corollary}
\newtheorem{lem}[thm]{Lemma}
\newtheorem*{lem*}{Lemma}
\newtheorem{prop}[thm]{Proposition}
\newtheorem*{prop*}{Proposition}
\newtheorem{conjecture}[thm]{Conjecture}
\newtheorem*{conjecture*}{Conjecture}
\newtheorem{fact}[thm]{Fact}
\newtheorem*{fact*}{Fact}

\theoremstyle{definition}
\newtheorem{defn}[thm]{Definition}
\newtheorem*{defn*}{Definition}
\newtheorem{example}[thm]{Example}
\newtheorem*{example*}{Example}
\newtheorem{problem}[thm]{Problem}
\newtheorem*{problem*}{Open Problem}
\newtheorem{xca}[thm]{Exercise}
\newtheorem*{xca*}{Exercise}
\newtheorem{sol}[thm]{Solution}
\newtheorem*{sol*}{Solution}

\theoremstyle{remark}
\newtheorem{rem}[thm]{Remark}
\newtheorem*{rem*}{Remark}
\newtheorem{idea}[thm]{Idea}
\newtheorem*{idea*}{Idea}
\newtheorem{claim}[thm]{Claim}
\newtheorem*{claim*}{Claim}


\usepackage{titlesec}
\usepackage{fancyhdr}
\pagestyle{fancy}
\usepackage{fix-cm}
\makeatletter
\newcommand\HUGE{\@setfontsize\Huge{38}{47}}
\makeatother

\titleclass{\part}{top}

\titleformat{\part}[display]
  {\normalfont\HUGE\sffamily}{\partname\ \thepart}{0pt}  
  {\titlerule\vskip2pt\titlerule\vskip20pt\HUGE\bfseries\filleft}
  
\titlespacing*{\part} {0pt}{30pt}{50pt}

\titleformat{\chapter}[display]
  {\normalfont\LARGE\sffamily}{\chaptertitlename\ \thechapter}{0pt}  
  {\titlerule\vskip2pt\titlerule\vskip20pt\LARGE\bfseries\filleft}

\titlespacing*{\chapter} {0pt}{20pt}{30pt}

\titleformat{\section}
  {\normalfont\Large\bfseries\sffamily}{\rule[.12ex]{8pt}{8pt}~\thesection}{0.5em}{}
  
\titleformat{\subsection}
  {\normalfont\large\bfseries\sffamily}{\rule[.12ex]{8pt}{8pt}~\thesubsection}{0.5em}{}
  
\fancyhf{}
\fancyhead[ER]{\footnotesize\sffamily\leftmark}
\fancyhead[OL]{\footnotesize\sffamily\nouppercase\rightmark}
\fancyhead[EL,OR]{\bfseries\thepage}
\usepackage{tikz}

% Define scaletikzpicturetowidth
% See https://tex.stackexchange.com/questions/6388/how-to-scale-a-tikzpicture-to-textwidth
\usepackage{environ}
\makeatletter
\newsavebox{\measure@tikzpicture}
\NewEnviron{scaletikzpicturetowidth}[1]{%
  \def\tikz@width{#1}%
  \def\tikzscale{1}\begin{lrbox}{\measure@tikzpicture}%
      \BODY
  \end{lrbox}%
  \pgfmathparse{#1/\wd\measure@tikzpicture}%
  \edef\tikzscale{\pgfmathresult}%
  \BODY
}
\makeatother


\usetikzlibrary{arrows,intersections}
\usetikzlibrary{shapes.misc}
\tikzset{cross/.style={cross out, draw=black, minimum size=2*(#1-\pgflinewidth), inner sep=0pt, outer sep=0pt},cross/.default={1pt}}



\makeatletter
\newcommand{\xleftrightarrow}[2][]{\ext@arrow 3359\leftrightarrowfill@{#1}{#2}}
\newcommand{\xdashrightarrow}[2][]{\ext@arrow 0359\rightarrowfill@@{#1}{#2}}
\newcommand{\xdashleftarrow}[2][]{\ext@arrow 3095\leftarrowfill@@{#1}{#2}}
\newcommand{\xdashleftrightarrow}[2][]{\ext@arrow 3359\leftrightarrowfill@@{#1}{#2}}
\def\rightarrowfill@@{\arrowfill@@\relax\relbar\rightarrow}
\def\leftarrowfill@@{\arrowfill@@\leftarrow\relbar\relax}
\def\leftrightarrowfill@@{\arrowfill@@\leftarrow\relbar\rightarrow}
\def\arrowfill@@#1#2#3#4{%
  $\m@th\thickmuskip0mu\medmuskip\thickmuskip\thinmuskip\thickmuskip
   \relax#4#1
   \xleaders\hbox{$#4#2$}\hfill
   #3$%
}

% alternative to arrow
\pgfarrowsdeclare{tonew}{tonew}
{
  \ifdim\pgfgetarrowoptions{tonew}=-1pt%
    \pgfutil@tempdima=0.84pt%
    \advance\pgfutil@tempdima by1.3\pgflinewidth%
    \pgfutil@tempdimb=0.21pt%
    \advance\pgfutil@tempdimb by.625\pgflinewidth%
  \else%
    \pgfutil@tempdima=\pgfgetarrowoptions{tonew}%
    \pgfarrowsleftextend{+-0.8\pgfutil@tempdima}%
    \pgfarrowsrightextend{+0.2\pgfutil@tempdima}%
  \fi%
}
{
  \ifdim\pgfgetarrowoptions{tonew}=-1pt%
    \pgfutil@tempdima=0.28pt%
    \advance\pgfutil@tempdima by.3\pgflinewidth%
    \pgfutil@tempdimb=0pt,%
  \else%
    \pgfutil@tempdima=\pgfgetarrowoptions{tonew}%
    \multiply\pgfutil@tempdima by 100%
    \divide\pgfutil@tempdima by 375%
    \pgfutil@tempdimb=0.4\pgflinewidth%
  \fi%
  \pgfsetdash{}{+0pt}
  \pgfsetroundcap
  \pgfsetroundjoin
  \pgfpathmoveto{\pgfpointorigin}
  \pgflineto{\pgfpointadd{\pgfpoint{0.75\pgfutil@tempdima}{0bp}}
                         {\pgfqpoint{-2\pgfutil@tempdimb}{0bp}}}
  \pgfusepathqstroke
  \pgfsetlinewidth{0.8\pgflinewidth}
  \pgfpathmoveto{\pgfpointadd{\pgfqpoint{-3\pgfutil@tempdima}{4\pgfutil@tempdima}}
                             {\pgfqpoint{\pgfutil@tempdimb}{0bp}}}
  \pgfpathcurveto
  {\pgfpointadd{\pgfqpoint{-2.75\pgfutil@tempdima}{2.5\pgfutil@tempdima}}
               {\pgfqpoint{0.5\pgfutil@tempdimb}{0bp}}}
  {\pgfpointadd{\pgfqpoint{0pt}{0.25\pgfutil@tempdima}}
               {\pgfqpoint{-0.5\pgfutil@tempdimb}{0bp}}}
  {\pgfpointadd{\pgfqpoint{0.75\pgfutil@tempdima}{0pt}}
               {\pgfqpoint{-\pgfutil@tempdimb}{0bp}}}
  \pgfpathcurveto
  {\pgfpointadd{\pgfqpoint{0pt}{-0.25\pgfutil@tempdima}}
               {\pgfqpoint{-0.5\pgfutil@tempdimb}{0bp}}}
  {\pgfpointadd{\pgfqpoint{-2.75\pgfutil@tempdima}{-2.5\pgfutil@tempdima}}
               {\pgfqpoint{0.5\pgfutil@tempdimb}{0bp}}}
  {\pgfpointadd{\pgfqpoint{-3\pgfutil@tempdima}{-4\pgfutil@tempdima}}
               {\pgfqpoint{\pgfutil@tempdimb}{0bp}}}
  \pgfusepathqstroke
}

% alias alternative to arrow
\pgfarrowsdeclarealias{<new}{>new}{tonew}{tonew}
\pgfsetarrowoptions{tonew}{-1pt}
\pgfkeys{/tikz/.cd, arrowhead/.default=-1pt, arrowhead/.code={
  \pgfsetarrowoptions{tonew}{#1},
}}
\makeatother


% \usepackage{url}
% \usepackage{xargs}
% \usepackage[pdftex,dvipsnames]{xcolor}
% \usepackage[colorinlistoftodos,prependcaption,textsize=tiny]{todonotes}
% \usepackage[small,bf]{caption}


% \usepackage{hyperref}
% \hypersetup{
    % colorlinks,
    % citecolor=red,
    % filecolor=red,
    % linkcolor=red,
    % urlcolor=red
% }


% \usepackage{fancybox}
% \usepackage{multirow}
% \usepackage{nicefrac}
% \usepackage{hyperref}


% \expandafter\def\expandafter\normalsize\expandafter{%
    % \normalsize
% %	    \setlength\abovedisplayskip{5pt}
    % \setlength\belowdisplayskip{5pt}
    % % \setlength\abovedisplayshortskip{5pt}
    % \setlength\belowdisplayshortskip{5pt}
% }

% \usepackage{tocloft}
% \setlength{\cftsecnumwidth}{2.8em}%



\makeatother

\usepackage{babel}
\begin{document}
In this book\footnote{In the UW course, only the optimization part is covered.},
we will study two topics involving convexity, namely optimization
and sampling. Given a multivariate, real-valued function $f$, (1)
how fast can we find a point that minimizes $f$? (2) how fast can
we sample a point according to the distribution with density defined
by $f$, e.g., proportional to $e^{-f}$? These problems are quite
closely connected, e.g., sampling from such distributions can be used
to find near-optimal points. These problems are intractable in full
generality, and have exponential (in dimension) complexity even under
smoothness assumptions.

Convexity and its natural extensions are a current frontier of tractable,
i.e., polynomial-time, computation. The assumption of convexity induces
structure in instances that makes them amenable to efficient algorithms.
For example, the local minimum of a convex function is a global minimum.
Convexity is maintained by natural operations such as intersection
(for sets) and addition (for functions). Perhaps less obvious, but
also crucial, is that convex sets can be approximated by ellipsoids
in various ways.

We will learn several techniques that lead us to some polynomial-time
algorithms for both problems and (nearly) linear-time algorithms for
the case $f$ is close to a quadratic function.

Although convex optimization has been studied since the 19th century\footnote{Augustin-Louis Cauchy introduced gradient descent in 1847.}
with many tight results emerging, there are still many basic open
problems. Here is an example:
\begin{problem*}
Given an $n\times n$ random $0/1$ matrix $A$ with $O(n)$ nonzero
entries, can we solve $Ax=b$ in $o(n^{2})$ time?
\end{problem*}
Computing the volume is an ancient problem, the early Egyptians and
Greeks developed formulas for specific shapes of interest. Unlike
convex optimization, even computing the volume of a convex body is
intractable, as we will see later. Nevertheless, there are efficient
randomized algorithms that can estimate the volume of convex bodies
to arbitrary accuracy in time polynomial in the dimension and the
desired accuracy. This extends to efficient algorithms for integrating
\emph{logconcave} functions, i.e., functions of the form $e^{-f}$
where $f$ is convex. The core ingredient is sampling in high dimension.
Sampling and volume computation will be the motivating problems for
the second part of this book. Again, many basic problems remain open.
To illustrate:
\begin{problem*}
Given a polytope defined by $\left\{ x:\,Ax\le b\right\} $, can we
estimate its volume to within a constant factor in nearly linear time?
\end{problem*}

\section{Why non-convex functions can be difficult}

Before discussing convex functions, we note that optimizing general
functions can be difficult. Consider the function
\[
f(x)=\begin{cases}
1 & \text{if }x\neq x^{*}\\
0 & \text{if }x=x^{*}
\end{cases}
\]
and suppose that we can only access the function by computing its
function value. This function $f$ always returns $1$ unless we know
$x^{*}$. Hence, one can prove that it takes infinitely many calls
to $f$ to find $x^{*}$.

This function is difficult to optimize, and not merely because of
its discontinuity. Similar functions can be constructed that are continuous.
Consider the function $f:B(0_{n},1)\rightarrow\R$ defined by
\begin{equation}
f(x)=\min(\|x-x^{*}\|_{2},\epsilon)\label{eq:f_lower_bound}
\end{equation}
where $B(0_{n},1)$ is the unit ball centered at the origin, $0_{n}$.
This function is $1$-Lipschitz and unless we query $f(x)$ with $x$
that is $\epsilon$-close to $x^{*}$, it will always return $\epsilon$.
Since the region where $f$ is not $\epsilon$ has volume $\frac{1}{\epsilon^{n}}$
times the volume of the unit ball, one can show that it takes $\Omega\left(\frac{1}{\epsilon^{n}}\right)$
calls to $f$ to find $x^{*}$. The following exercise asks you to
prove that this bound is tight.
\begin{xca}
Show that if $f$ is $1$-Lipschitz on $B(0_{n},1)$, we can find
$x$ such that $f(x)-\min_{x}f(x)\leq\epsilon$ by calling $f(x)$
at $O(\frac{1}{\epsilon})^{n}$ points. 
\end{xca}

Thus $O(1/\epsilon)^{n}$ is the best possible bound for optimizing
general $1$-Lipschitz functions. Similar constructions can also be
made for infinitely differentiable functions. We note that it is easy
to find local minima for all the functions above. In Section \ref{sec:Gradient-Descent},
we will show that it is easy to find an approximate local minimum
of a continuously differentiable function.

\section{Why is Convexity Useful? Linear Separability!}

In the last section, we saw that general functions are difficult to
optimize because we can only find the minimum via exhaustive search.
Now we define convex sets, convex functions and convex problems. One
benefit of convexity is that it enables binary search.
\begin{defn}
A set $K$ in $\R^{n}$ is \emph{convex} if for every pair of points
$x,y\in K$, we have $[x,y]\subseteq K$.
\end{defn}

\begin{defn}
\label{def:convex_func}A function $f:\R^{n}\rightarrow\R\cup\{+\infty\}$
is \emph{convex} if for any $\lambda\in[0,1]$, we have 
\[
f(\lambda x+(1-\lambda)y)\le\lambda f(x)+(1-\lambda)f(y).
\]

\begin{figure}
    \begin{minipage}[t]{0.47\linewidth}         \centering         \begin{scaletikzpicturetowidth}{\linewidth}             \begin{tikzpicture}[label distance = 2pt, scale = \tikzscale]                 \draw[-, thick, red] (-1.5, 0) -- (3.5, 0);                 \draw[-, thick, {black!50!green}] (-1, 0.75) -- (2, 1.5);                 \draw[blue] (-1.5, 1.0625) node[label=left:$f$] {};                 \draw[domain=-1.5:3, smooth, variable=\x, blue] plot ({\x},{\x * \x / 4 + 0.5});                 \draw node[mark size=1.5pt, color={black!50!green}] at (-1, 0.75) {\pgfuseplotmark{*}};                 \draw node[mark size=1.5pt, color={black!50!green}] at (2, 1.5) {\pgfuseplotmark{*}};                 \draw node[mark size=1.5pt, color={black!50!green}] at (0.5, 0.5625) {\pgfuseplotmark{*}};                 \draw node[mark size=5pt, color={black!50!green}, thick, label={[black!50!green]below:$x$}] at (-1, 0) {\pgfuseplotmark{|}};                 \draw node[mark size=5pt, color={black!50!green}, thick, label={[black!50!green]below:$y$}] at (2, 0) {\pgfuseplotmark{|}};                 \draw node[mark size=5pt, color={black!50!green}, thick] at (0.5, 0) {\pgfuseplotmark{|}};             \end{tikzpicture}         \end{scaletikzpicturetowidth}         % \caption*{Convex function}     
\end{minipage}
 \hfill     
\begin{minipage}[t]{0.47\linewidth}         \centering         \begin{scaletikzpicturetowidth}{\linewidth}             \begin{tikzpicture}[label distance = 2pt, scale = \tikzscale]                 \draw[-, thick, red] (-4, 0) -- (5, 5.196152423);                 \draw node[mark size=1.5pt, color={black!50!green}, thick, label={[black!50!green]left:$y$}] at (0, 3.464101615) {\pgfuseplotmark{*}};                 \draw[-, thick, blue] (0, 0) -- (1, 1.732050808) -- (3, 1.732050808) -- (4, 0) -- (3, -1.732050808) -- (1, -1.732050808) -- (0, 0);                 \draw[blue] (1.5, 0) node[label=$K$] {};             \end{tikzpicture}         \end{scaletikzpicturetowidth}         % \caption*{Convex set}     
\end{minipage}

\caption{Convex function; Convex set\label{fig:convex}}
\end{figure}
\end{defn}

\begin{defn}
An optimization problem $\min_{x\in K}f(x)$ is convex if $K$ and
$f$ are convex.
\end{defn}

Any point not in a convex set can be separated by a hyperplane from
the set. We will see in Chapter \ref{chap:Elimination} that separating
hyperplanes allow us to do binary search to find a point in a convex
set. This is the basis of all polynomial-time algorithms for optimizing
general convex functions. We will explore these binary search algorithms
in Chapter \ref{chap:Elimination}.
\begin{thm}[Hyperplane separation theorem]
\label{thm:sep}Let $K$ be a closed convex set in $\Rn$ and $y\notin K$.
There is a non-zero $\theta\in\Rn$ such that 
\[
\left\langle \theta,y\right\rangle >\max_{x\in K}\left\langle \theta,x\right\rangle .
\]
\end{thm}

\begin{proof}
Let $x^{*}$ be a point in $K$ closest to $y$, namely $x^{*}\in\arg\min_{x\in K}\norm{x-y}_{2}^{2}$
(such a minimizer always exists for closed convex sets; this is sometimes
called Hilbert's projection theorem). Using convexity of $K$, for
any $x\in K$ and any $0\leq t\leq1$, we have that $(1-t)x^{*}+tx\in K$
and hence
\[
\norm{y-(1-t)x^{*}-tx}_{2}^{2}\geq\min_{x\in K}\norm{y-x}_{2}^{2}=\norm{y-x^{*}}_{2}^{2}.
\]
Expanding the LHS, we have
\begin{align*}
\norm{y-(1-t)x^{*}-tx}_{2}^{2} & =\norm{y-x^{*}+t(x^{*}-x)}_{2}^{2}\\
 & =\norm{y-x^{*}}_{2}^{2}+2t\left\langle y-x^{*},x^{*}-x\right\rangle +t^{2}\norm{x^{*}-x}_{2}^{2}.
\end{align*}
Canceling the term $\norm{x^{*}-y}_{2}^{2}$ and dividing both sides
by $t$, 
\[
2\left\langle y-x^{*},x^{*}-x\right\rangle +t\norm{x^{*}-x}_{2}^{2}\ge0.
\]
Taking $t\rightarrow0^{+}$, we have that
\begin{equation}
\left\langle y-x^{*},x^{*}-x\right\rangle \geq0\text{ for all }x\in K.\label{eq:seq_1}
\end{equation}
Taking $\theta=y-x^{*}$ and using (\ref{eq:seq_1}), for all $x\in K$,
we have that
\begin{align*}
\left\langle \theta,y-x\right\rangle  & =\left\langle \theta,y-x^{*}\right\rangle +\left\langle y-x^{*},x^{*}-x\right\rangle \\
 & =\|\theta\|^{2}+\left\langle y-x^{*},x^{*}-x\right\rangle \\
 & >0
\end{align*}
where we used that $y\notin K$ and hence $\|\theta\|^{2}>0$.
\end{proof}
Theorem \ref{thm:sep} shows that a polytope (a finite intersection
of halfspaces) is essentially as general as a convex set.
\begin{cor}
\label{cor:convex_function_intersection}Any closed convex set $K$
can be written as the intersection of halfspaces as follows
\[
K=\bigcap_{\theta\in\Rn}\left\{ x:\ \left\langle \theta,x\right\rangle \leq\max_{y\in K}\left\langle \theta,y\right\rangle \right\} .
\]
In other words, any convex set is a limit of a sequence of polyhedra.
\end{cor}

\begin{proof}
Let $L\defeq\bigcap_{\theta\in\Rn}\left\{ x:\ \left\langle \theta,x\right\rangle \leq\max_{y\in K}\left\langle \theta,y\right\rangle \right\} $.
Since $K\subset\left\{ x:\ \left\langle \theta,x\right\rangle \leq\max_{y\in K}\left\langle \theta,y\right\rangle \right\} $,
we have $K\subset L$. 

For any $x\notin K$, Theorem \ref{thm:sep} shows that there is a
$\theta$ such that $\theta^{\top}x>\max_{y\in K}\theta^{\top}y$.
Hence, we have $x\notin L$ and hence $L\subset K$.
\end{proof}
This shows that convex optimization is related to linear programs
(optimize linear functions over polytopes) as follows:
\[
\min_{x\in K}f(x)=\min_{(x,y)\in\{x\in K,y\in\R:y\geq f(x)\}}y
\]
where the set $\{x\in K,y\in\R:y\geq f(x)\}$ then can be approximated
by intersection of halfspaces, namely $\{A\binom{x}{y}\leq b\}$ for
some matrix $A\in\R^{m\times(n+1)}$ and vector $b\in\R^{m}$ with
$m\rightarrow+\infty$.

Similar to convex sets, we have a separation theorem similar to Theorem
\ref{thm:sep} for convex functions. This shows that one can use binary
search to find minimum of convex functions. (See Chapter \ref{chap:Elimination}.)
\begin{thm}
\label{thm:first_order}Let $f\in\mathcal{C}^{1}(\Rn)$ be convex.
Then, 
\begin{equation}
f(y)\geq f(x)+\nabla f(x)^{\top}(y-x)\text{ for all }x,y\in\Rn\label{eq:f_lower}
\end{equation}
\end{thm}

\begin{proof}
Fix any $x,y\in\Rn$. Let $g(t)=f((1-t)x+ty)$. Since $f$ is convex,
so is $g$ over $[0,1]$. Then, we have
\[
g(t)\leq(1-t)g(0)+tg(1)
\]
which implies that
\[
g(1)\geq g(0)+\frac{g(t)-g(0)}{t}.
\]
Taking $t\rightarrow0^{+}$, we have that $g(1)\geq g(0)+g'(0)$.
In other words, using the chain rule for derivatives, we have $g'(0)=\left\langle \nabla f(x),y-x\right\rangle $
and hence
\[
f(y)\geq f(x)+\left\langle \nabla f(x),y-x\right\rangle .
\]
\end{proof}
This theorem shows that $\nabla f(x)=0$ (local minimum) implies $x$
is a global minimum.
\begin{thm}[Optimality condition for unconstrained problems]
\label{thm:optimality_condition}Let $f\in\mathcal{C}^{1}(\Rn)$
be convex. Then, $x\in\R^{n}$ is a minimizer of $f(x)$ if and only
if $\nabla f(x)=0$.
\end{thm}

\begin{proof}
If $\nabla f(x)\neq0$, then 
\[
f(x-\varepsilon\nabla f(x))=f(x)-\varepsilon\norm{\nabla f(x)}_{2}^{2}+O(\varepsilon^{2})\norm{\nabla f(x)}_{2}^{2}<f(x)
\]
for small enough $\varepsilon$. Hence, such a point cannot be the
minimizer.

On the other hand, if $\nabla f(x)=0$, Theorem \ref{thm:first_order}
shows that
\[
f(y)\geq f(x)+\nabla f(x)^{\top}(y-x)=f(x)\text{ for all }y.
\]
\end{proof}
We note that the proof above is in fact a constructive proof. If $x$
is not a minimum, it suggests a point that has better function value.
This will be the discussion of an upcoming section. For continuous
convex functions, there is a weaker notion of gradient called sub-differential,
which is a set instead of a vector. Both theorems above holds with
gradients replaced by sub-differentials.

\section{Convex Problems are Everywhere!}

In this section, we give a few examples of convex problems to illustrate
the wide applicability of convex optimization. 

\subsection{Minimum Cost Flow Problem (Computer Science)}

The min cost flow problem has lots of applications such as route planning,
airline scheduling, image segmentation, recommendation systems, etc.
In this problem, we are given a graph $G=(V,E)$ with $m\defeq|E|$
edges and $n\defeq|V|$ vertices. Each edge $e\in E$ has capacity
$u_{e}>0$ and cost $c_{e}$. The problem is to minimize the total
cost of sending $d$ amount of flow from a source vertex $s\in V$
to a sink vertex $t\in V$. Formally, the problem can be written as
an optimization problem $\min_{f\in\R^{|E|}}\sum_{e\in E}c_{e}\cdot f_{e}$
subject to the constraints:
\begin{itemize}
\item Capacity constraints: $0\leq f_{e}\leq u_{e}$ for all $e\in E$.
\item Flow conservation: $\sum_{e\text{ enters }v}f_{e}=\sum_{e\text{ leaves }v}f_{e}$
for all $v\in V\setminus\left\{ s,t\right\} $.
\item Demand: $\sum_{e\text{ enters }t}f_{e}=\sum_{e\text{ leaves }s}f_{e}=d$.
\end{itemize}
To check this is a convex problem, we note that the objective function
is a linear function $c^{\top}f$ which is convex. The domain is the
intersection of three sets (the three set of equations above). The
first set is a scaled hypercube, the second and last set is a linear
subspace. All of them are convex and so is their intersection. Therefore,
this is a convex problem.

\subsection{Linear Programs (Operation Research/Economics)}

Consider the diet problem: find a cheapest diet plan that satisfies
the nutrients requirements. Formally, suppose there are $n$ different
foods and $m$ different nutrients. The food $i$ has unit cost $c_{i}$,
$a_{1i}$ unit of nutrient $1$, $a_{2i}$ unit of nutrient $2$,
$\cdots$. Furthermore, every day we need $b_{j}$ unit of the nutrient
$j$. Then, the problem is simply find an assignment $x$ such that
\[
\min_{x\geq0,Ax\geq b}c^{\top}x
\]
where $c\in\Rn$ is the cost vector, $b\in\Rm$ is the intake requirement
vector and $A\in\R^{m\times n}$ is the matrix of nutrients contents
of each food. Unfortunately, Nobel Laureate Stigler showed that \cite{stigler1945cost}
the optimal meal is evaporated milk, cabbage, dried navy beans, and
beef liver.

Joking aside, both the diet problem and the minimum cost flow problem
can be reformulated into the form
\begin{equation}
\min_{Ax=b,x\geq0}c^{\top}x\label{eq:LP_standard}
\end{equation}
for some vectors $c,b$ and some matrix $A$. These problems are called
linear programs and have many applications in resource allocation.
Special cases of linear programs are also of great interest; for example
the diet problem is a packing/covering LP.
\begin{xca}
Show that both the minimum cost flow problem and the diet problem
can be written as (\ref{eq:LP_standard}). Also, show that (\ref{eq:LP_standard})
is a convex problem.
\end{xca}


\subsection{Logistic Regression (Machine Learning)}

Consider the problem of predicting the likelihood of getting diabetes
in the future. Suppose we have collected many examples $\{(x_{i},y_{i})\}_{i=1}^{n}$
where $x_{i}\in\R^{d}$ represents the features of a person and $y_{i}\in\{\pm1\}$
represents whether that person gets diabetes. For example, the feature
vector can be (age, weight, height, BMI, fasting glucose level, ...).
The features in the vector may be redundant and the purpose of extra
variables is to make linear functions expressive enough to be able
to classify. In particular, we assume that there is a vector $\theta$
such that for most $i$, we have $\langle x^{i},\theta\rangle<0$
if $y_{i}=1$ and $\langle x^{i},\theta\rangle>0$ if $y_{i}=-1$
. The error of the vector $\theta$ is
\[
\frac{1}{n}\sum_{i=1}^{n}1_{y_{i}\langle x^{i},\theta\rangle>0}.
\]
This function is not convex in $\theta$. More generally, one considers
the objective function (to be minimized over $\theta$)
\begin{equation}
R(\theta)=\frac{1}{n}\sum_{i=1}^{n}f(y_{i}\langle x^{i},\theta\rangle)+\lambda\|\theta\|_{1}\label{eq:logistic_regression}
\end{equation}
where $f$ is some function such that $f(z)$ is large when $z$ is
positive and large and $f(z)=0$ when $z$ is highly negative, and
$\lambda\|\theta\|_{1}$ is a \emph{regularization} term to make sure
$\theta$ is bounded. One popular function is $f(z)=\log(1+e^{z})$,
and this problem is called logistic regression. In the section \ref{sec:Checking-Convexity},
we prove that the function (\ref{eq:logistic_regression}) is indeed
convex when $f(z)=\log(1+e^{z})$.

\subsection{Minimal Surface (Physics)}

A surface $M\subset\R^{3}$ is a minimal surface if it has the minimum
surface area among all surfaces with the same boundary. These surfaces
appear naturally and has been studied extensively not just for $\R^{3}$
but also for different manifold. For simplicity, we consider the case
that the surface is parameterized by 
\[
M=\{(x,y,f(x,y)):0\leq x\leq1,0\leq y\leq1\}.
\]
In this case, $f$ is a minimal surface if
\[
f=\argmin_{\text{feasible }g}\text{SurfaceArea}(g)
\]
where we say $g$ is feasible if $g(x,y)=f(x,y)$ for all $x,y$ on
the boundary of $[0,1]^{2}$. One natural question (called Plateau's
problem) is to find a minimal surface with a given boundary. For this
particular case we consider, we can simply use convex optimization.
Note that the constraint ($g$ is feasible) is exactly a linear subspace
on the space of functions on $[0,1]^{2}$. Furthermore, the objective
is convex by using the fact that
\[
\text{SurfaceArea}(f)=\int_{0}^{1}\int_{0}^{1}\sqrt{(\frac{\partial f(x,y)}{\partial x})^{2}+(\frac{\partial f(x,y)}{\partial y})^{2}+1}dxdy.
\]

\begin{xca}
Show that surface area is convex by using the definition.

Calculus of variations is an area in mathematics studies the optimization
on function spaces and there are many common theorems between this
and convex optimization.
\end{xca}


\section{Examples of Convex Sets and Functions}

There are many important convex sets and here we only list some that
appear in this course. One of the most important classes of convex
functions comes from convex sets.
\begin{defn}
For a convex set $K$, we define the indicator function of $K$ by
\[
\delta_{K}(x)=\begin{cases}
0 & \text{if }x\in K\\
+\infty & \text{otherwise}
\end{cases}.
\]
\end{defn}

We can also construct convex sets by convex functions. We let the
domain $\dom f\defeq\{x\in\Rn:\ f(x)<+\infty\}$. The definition of
a convex function shows that $\dom f$ is a convex set if $f$ is
a convex function. Alternatively, by looking at the set of points
above the graph of the function, we obtain a convex set called an
epigraph.
\begin{defn}
\label{def:epigraph}The \emph{epigraph} of $f:\Rn\rightarrow\R$
is $\epi f\defeq\{(x,t)\subset\Rn\times\R:\ t\geq f(x)\}$.

A function $f$ is convex if and only if  $\epi f$ is a convex set.
\end{defn}

\begin{figure}
 \begin{minipage}[t]{0.47\linewidth}         \centering         \begin{scaletikzpicturetowidth}{\linewidth}             \begin{tikzpicture}[label distance = 2pt, scale = \tikzscale]                 \draw[-, thick, {black!50!green}] (-1.5, 1.375) -- (-1.5, 3);                 \draw[-, thick, {black!50!green}] (-1, 0.75) -- (-1, 3);                 \draw[-, thick, {black!50!green}] (-0.5, 0.375) -- (-0.5, 3);                 \draw[-, thick, {black!50!green}] (0, 0.25) -- (0, 3);                 \draw[-, thick, {black!50!green}] (0.5, 0.375) -- (0.5, 3);                 \draw[-, thick, {black!50!green}] (1, 0.75) -- (1, 3);                 \draw[-, thick, {black!50!green}] (1.5, 1.375) -- (1.5, 3);                 \draw[black!50!green] (1.5, 2.5) node[label=right:$\epi(f)$] {};                 \draw[blue] (2, 2) node[label=right:$f$] {};                 \draw[domain=-2:2, smooth, variable=\x, blue] plot ({\x},{\x * \x / 2});             \end{tikzpicture}         \end{scaletikzpicturetowidth}         % \caption*{Epigraph of $f$}     
\end{minipage}     
\hfill     
\begin{minipage}[t]{0.47\linewidth}         \centering         \begin{scaletikzpicturetowidth}{\linewidth}             \begin{tikzpicture}[label distance = 2pt, scale = \tikzscale]                 \draw[-, thick, red] (0, 0) -- (10, 0);                 \draw[blue] (1, 4) node[label=below left:$f$] {};                 \draw[-, thick, blue] (0, 5) -- (1, 5) -- (1, 4) -- (2, 4) -- (3, 3) -- (4, 3);                 \draw[thick, blue] plot [smooth, tension=2] coordinates{(4, 3) (5, 2) (6, 3)};                 \draw[-, thick, blue, rounded corners=7pt] (6, 3) -- (7, 3) -- (7.5, 4) -- (8.5, 4) -- (9, 5) -- (10, 5);             \end{tikzpicture}         \end{scaletikzpicturetowidth}         
% \caption*{Quasiconvex function}     
\end{minipage}\caption{Epigraph of $f$; quasiconvex function\label{fig:rel-1-1}}
\end{figure}

This characterization shows that $\min_{x}f(x)$ is the same as $\min_{(t,x)\in\epi f}t$.
Therefore, convex optimization is the same as optimizing a linear
function over a convex set. Another important feature of a convex
set is the following.
\begin{fact}
Any level set $\{x\in\Rn:\ f(x)\leq t\}$ of a convex function $f$
is convex.
\end{fact}

In particular, this shows that the set of minimizers is connected.
Therefore, any local minimum is a global minimum. We note that the
converse of the fact above is not true. A function is \emph{quasiconvex}
if every level set is convex. For example, Fig. \ref{fig:rel-1-1}
shows a function that is quasiconvex but not convex.

Finally, we note that many operations preserve convexity. Here is
an example.
\begin{xca}
Given a matrix $A$, a vector $b$, positive scalars $t_{1},t_{2}\geq0$,
convex functions $f_{1}$ and $f_{2}$, then $g(x)=t_{1}f_{1}(Ax+b)+t_{2}f_{2}(x)$
is convex.
\end{xca}

Here are some convex sets and functions. In Section \ref{sec:Checking-Convexity},
we illustrate how to check convexity.
\begin{example*}
Convex sets: polyhedron $\{x:\,Ax\leq b\}$, polytope $\conv\left(\left\{ v_{1},\ldots,v_{m}\right\} \right)$
with $v_{1},\ldots,v_{m}\in\R^{n}$, ellipsoid $\{x:\,x^{\top}Ax\leq1\}$
with $A\succeq0$, positive semidefinite cone $\{X\in\R^{n\times n}\text{ : }X\succeq0\}$,
norm ball $\{x:\ \norm x_{p}\leq1\}$ for all $p\geq1$.
\end{example*}
%
\begin{example*}
Convex functions: $x$, $\max(x,0)$, $e^{x}$, $|x|^{a}$ for $a\geq1$,
$-\log(x)$, $x\log x$, $\norm x_{p}$ for $p\geq1$, $(x,y)\rightarrow\frac{x^{2}}{y}$
(for $y>0$), $A\rightarrow-\log\det A$ over PSD matrices $A$, $(x,Y)\rightarrow x^{\top}Y^{-1}x$
(for $Y\succ0$), $\log\sum_{i}e^{x_{i}}$, $(\prod_{i}^{n}x_{i})^{\frac{1}{n}}$.
\end{example*}
\begin{xca}
Show that the above sets and functions are all convex.
\end{xca}


\section{Checking Convexity\label{sec:Checking-Convexity}}

It is often cumbersome to check if a function is convex via (\ref{def:convex_func}).
The major benefit of that definition is that it works for non-differentiable
functions and infinite dimensional space. However, when a function
is twice differentiable, one can check the convexity by simply checking
if the Hessian is positive semi-definite. (Recall the definition of
$A\succeq0$ in \ref{def:PSD}.)

To show this, we first prove the following second-order Taylor theorem.
\begin{lem}
\label{lem:second_order}For any $f\in\mathcal{C}^{2}(\R^{n})$, and
any $x,y\in\R^{n}$, there is a $z\in[x,y]$ s.t. 
\[
f(y)=f(x)+\nabla f(x)^{\top}(y-x)+\frac{1}{2}(y-x)^{\top}\nabla^{2}f(z)(y-x).
\]
\end{lem}

\begin{proof}
Let $g(t)=f((1-t)x+ty)$. Taylor expansion (Theorem \ref{thm:taylor_expansion})
shows that
\[
g(1)=g(0)+g'(0)+\frac{1}{2}g''(\zeta)
\]
where $\zeta\in[0,1]$. To see the result, note that $g(0)=f(x)$,
$g'(0)=\nabla f(x)^{\top}(y-x)$ and $g''(\zeta)=(y-x)^{\top}\nabla^{2}f((1-\zeta)x+\zeta y)(y-x)$. 
\end{proof}
Now, we show that $f$ is convex if and only if $\nabla^{2}f(x)\succeq0$
for all $x$.
\begin{thm}
\label{thm:equ_convex}Let $f\in\mathcal{C}^{2}(\Rn)$. Then, the
following are equivalent:
\begin{enumerate}
\item $f$ is convex.
\item $f(y)\geq f(x)+\nabla f(x)^{\top}(y-x)\text{ for all }x,y\in\Rn$.
\item $\nabla^{2}f(x)\succeq0$ for all $x\in\Rn$.
\end{enumerate}
\end{thm}

\begin{proof}
We have proved (1) implies (2) in Theorem \ref{thm:first_order}.

Suppose (2) holds. Then, for any $x,h\in\Rn$
\[
f(x+th)\geq f(x)+t\nabla f(x)^{\top}h.
\]
By Taylor expansion (Lemma \ref{lem:second_order}), we have that
\[
f(x+th)=f(x)+t\nabla f(x)^{\top}h+\frac{t^{2}}{2}h^{\top}\nabla^{2}f(z)h
\]
where $z\in[x,x+th]$. By comparing two equations, we have that $h^{\top}\nabla^{2}f(z)h\geq0$.
Taking $t\rightarrow0$, we have $z\rightarrow x$ and hence $\nabla^{2}f(z)\rightarrow\nabla^{2}f(x)$.
Therefore, we have that
\[
h^{\top}\nabla^{2}f(x)h\geq0
\]
for all $x$ and $h$. Hence, this gives (3).

Suppose (3) holds. Fix $x,y\in\Rn$. Consider the function
\[
g(\lambda)=f(\lambda x+(1-\lambda)y)-\lambda f(x)-(1-\lambda)f(y).
\]
Consider $\lambda^{*}=\argmax_{\lambda\in[0,1]}g(\lambda)$. If $\lambda^{*}$
is either $0$ or $1$, then we have $g(\lambda^{*})=0$. Otherwise,
by Taylor's theorem, there is a $\zeta\in[\lambda^{*},1]$ such that
\begin{align*}
g(1) & =g(\lambda^{*})+g'(\lambda^{*})(1-\lambda^{*})+\frac{1}{2}g''(\zeta)(1-\lambda^{*})^{2}\\
 & =g(\lambda^{*})+\frac{1}{2}g''(\zeta)(1-\lambda^{*})^{2}
\end{align*}
where we used that $g'(\lambda^{*})=0$. Note that 
\begin{align*}
g'(\zeta) & =\nabla f(\zeta x+(1-\zeta)y)^{\top}(x-y)-f(x)+f(y),\\
g''(\zeta) & =(x-y)^{\top}\nabla^{2}f(\zeta x+(1-\zeta)y)(x-y).
\end{align*}
By the assumption (3), we have that $g''(\zeta)\geq0$ and hence $0=g(1)\geq g(\lambda^{*})$.
Hence, in both cases, $\max_{\lambda\in[0,1]}g(\lambda)=g(\lambda^{*})\leq0$.
This gives (1).
\end{proof}
Now, as an example, we prove that the function (\ref{eq:logistic_regression})
is convex.
\begin{example}
The function (\ref{eq:logistic_regression}) is convex for $f(z)=\log(1+\exp(z))$.
\end{example}

\begin{proof}
We write $R(\theta)=R_{1}(\theta)+R_{2}(\theta)$ where $R_{1}(\theta)=\frac{1}{n}\sum_{i=1}^{n}f(y_{i}\cdot\left\langle x^{i},\theta\right\rangle )$
and $R_{2}(\theta)=\lambda\|\theta\|_{1}$. It is easy to check $R_{2}$
is convex. So, it suffices to prove $R_{1}$ is convex. Now, we use
Theorem \ref{thm:equ_convex} to prove that $R_{1}$ is convex. Note
that
\begin{align*}
\nabla R_{1}(\theta) & =\frac{1}{n}\sum_{i=1}^{n}f'(\left\langle x^{i},\theta\right\rangle )x^{i},\\
\nabla^{2}R_{1}(\theta) & =\frac{1}{n}\sum_{i=1}^{n}f''(\left\langle x^{i},\theta\right\rangle )x^{i}(x^{i})^{\top}.
\end{align*}
Since $x^{i}(x^{i})^{\top}\succeq0$, it suffices to prove that $f''(\left\langle x^{i},\theta\right\rangle )\geq0$.
This follows from the calculation: $f'(z)=\frac{\exp(z)}{1+\exp(z)}=1-\frac{1}{1+\exp(z)}$
and $f''(z)=\frac{\exp(z)}{(1+\exp(z))^{2}}\geq0$.
\end{proof}

\section{Logconcave Functions\label{sec:Logconcave-functions}}
\begin{defn}
A function $f:\R^{n}\rightarrow\R_{+}$ is \emph{logconcave} if $\log f$
is concave, i.e., $f$ is nonnegative and for any $\lambda\in[0,1]$,
we have $f(\lambda x+(1-\lambda)y)\ge f(x)^{\lambda}f(y)^{1-\lambda}$.
\end{defn}

\begin{example}
The indicator function of a convex set $1_{K}(x)=\begin{cases}
1 & \text{if }x\in K\\
0 & \text{otherwise}
\end{cases}$ is logconcave.
\end{example}

\begin{lem}[Dinghas; Pr�kopa; Leindler]
\label{lem:marginal} The product, minimum and convolution of two
logconcave functions is also logconcave; in particular, any linear
transformation or any marginal of a logconcave density is logconcave;
the distribution function of any logconcave density is logconcave.
\end{lem}

We next describe the basic theorem underlying the above properties.
We will see their proofs in a later chapter.
\begin{thm}[Pr�kopa-Leindler]
\label{thm:Prekopa-Leindler}Fix $\lambda\in[0,1].$ Let $f,g,h:\R^{n}\rightarrow\R_{+}$
be functions satisfying $h(\lambda x+(1-\lambda)y)\ge f(x)^{\lambda}g(x)^{1-\lambda}$
for all $x\in\R^{n}$. Then,
\[
\int_{\R^{n}}h\ge\left(\int_{\R^{n}}f\right)^{\lambda}\left(\int_{\R^{n}}g\right)^{1-\lambda}.
\]
\end{thm}

An equivalent version of the lemma for sets in $\R^{n}$ is often
useful. By a \emph{measurable }set below, we mean Lebesgue measurable,
which coincides with the definition of volume (for an axis aligned
box, it is the product of the axis lengths; for any other set, it
is the limit over increasingly finer partitions into boxes, of the
sum of volumes of boxes that intersect the set).
\begin{thm}[Brunn-Minkowski]
\label{thm:Brunn-Minkowski}For any $\lambda\in[0,1]$ and measurable
sets $A,B\subset\R^{n}$, we have 
\[
\vol(\lambda A+(1-\lambda)B)^{1/n}\ge\lambda\vol(A)^{1/n}+(1-\lambda)\vol(B)^{1/n}.
\]
\end{thm}

An immediate consequence of the first theorem above is that any one-dimensional
marginal distribution of a convex body is logconcave; the second implies
that it is in fact $(1/(n-1))$-concave (a function is $s$-concave
if $f^{s}$ is concave) if the body is in $\R^{n}$.
\begin{xca}
Prove both corollaries just mentioned.
\end{xca}

\begin{example}
We give one example problem related to Bayesian inference. Suppose
we have a signal $\theta=\left(\theta_{1},\theta_{2},\cdots,\theta_{n}\right)$
and that we can take a measurement $y_{i}$ of $\theta_{i}$. The
measurement only incurs unbiased Gaussian noise, i.e., $y_{i}=\theta_{i}+\epsilon_{i}$
where $\epsilon_{i}\sim N(0,1)$. The question is to recover the signal
$\theta$ using $y$. Without any prior on $\theta$, the only sensible
recovery is $\theta=y$. With a prior, one can apply Bayes' theorem:
\[
\P(\theta|y)=\frac{\P(y|\theta)\P(\theta)}{\P(y)}.
\]

The Bayesian choice is to find $\theta$ with maximum likelihood,
namely $\theta=\argmax_{\theta}\log\P(\theta|y)=\argmin_{\theta}-\log\P(\theta|y)$.

Using the noise assumption, we have that 
\[
\P(y|\theta)\varpropto\exp(-\frac{1}{2}\sum_{i}(y_{i}-\theta_{i})^{2}).
\]
Now, say we know the signal is smooth and we model the prior as $\P(\theta)\varpropto\exp(-\lambda\sum_{i}(\theta_{i}-\theta_{i+1})^{2})$
where $\lambda$ controls how smooth the signal is. Hence, 
\[
-\log\P(\theta|y)=c+\sum_{i}(y_{i}-\theta_{i})^{2}+\lambda\sum_{i}(\theta_{i}-\theta_{i+1})^{2}.
\]
Since each term in the function above is convex, so is the whole formula.
Hence, the recovery question becomes a convex optimization problem
\[
\min_{\theta}\sum_{i}(y_{i}-\theta_{i})^{2}+\lambda\sum_{i}(\theta_{i}-\theta_{i+1})^{2}.
\]

When we recover a signal, we want to know how confident we are because
there are many choices of $\theta$ that could explain the same measurement
$y$. One way to do this is to sample multiple $\theta\varpropto\P(\theta|y)$
and compute the empirical variance or other statistics. Note that
\[
\P(\theta|y)\varpropto e^{-\sum_{i}(y_{i}-\theta_{i})^{2}-\lambda\sum_{i}(\theta_{i}-\theta_{i+1})^{2}}
\]
which is a logconcave distribution. Therefore, one can study the signal
and quality of signal recovery via logconcave sampling.
\end{example}


\end{document}
